\chapter{Variations}

Other Topics, Related Topics

\section{Static Mastermind}

\section{Dynamic Mastermind}

In the standard Mastermind game, all that the code-maker does is to set up a secret at the beginning of the game, and then \emph{passively} responds to the guesses made by the code-maker. The role played by the code-maker is rather boring.

To make the code-maker's role more interesting (and challenging), in a \emph{dynamic} Mastermind game, introduced by Bestavros and Belal \cite{bestavros86}, the codemaker is allowed to silently change the secret during the course of the game, as long as the secret is consistent with all the guesses and feedbacks so far. 

For example, suppose the code-maker initially holds the secret 1234. Suppose that in the first round, the code-breaker makes the guess 1234 outright. Under the standard rules, the code-maker is obliged to respond with 4A0B and surrender the game. However, under the \emph{dynamic} rules, the code-maker is allowed to silently change the secret to, say, 3456, and reply with the feedback 0A2B. The game continues until the code-maker finally reveals the secret.

It is easy to see that under either the standard rules or the dynamic rules, the code-maker will eventually lose because there are a finite number of codewords. However, under the dynamic rules, the code-maker is able to prolong the game for more steps. 

In fact, it can be shown that when both sides play optimally, the game will require exactly [7??] rounds to finish. That is, after six rounds of guesses and responses, there is only one codeword left that conforms to all the constraints so far; the code-breaker will then guess this codeword and the code-maker has to respond with 4A0B.

[Show this using exhaustive search]

[Compare different combinations of heuristic strategies used by the code-maker and code-breaker]


\section{Mastermind with a lie}

