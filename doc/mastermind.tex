%\documentclass[8pt,a5paper]{article}
\documentclass{report}
\usepackage{amsmath}
\usepackage{amsthm}
\usepackage{hyperref}
\hypersetup{
	pdftitle={Mastermind Strategies},
	pdfkeywords={Mastermind, heuristic},
	colorlinks=true,
	pdfstartview={FitH},
	linkcolor=black,
	citecolor=blue,
	urlcolor=black,
}
%\usepackage[small,textfont=it]{caption}
\usepackage[small]{caption}

\setlength{\parskip}{1ex plus 0.5ex minus 0.2ex}

\newcommand{\given}{\,|\,}
\newcommand{\minmax}{\texttt{min-max}}
\newcommand{\minavg}{\texttt{min-avg}}
\newcommand{\maxent}{\texttt{max-entropy}}
\newcommand{\maxpar}{\texttt{max-parts}}
\newcommand{\cw}[1]{\texttt{#1}}
\newcommand{\fb}[2]{\texttt{{#1}A{#2}B}}
%\newcommand{\u}[1]{\underline{#1}}

\theoremstyle{definition}
\newtheorem*{definition}{Definition}
%\newtheorem{definition}{Definition}
\newtheorem*{note}{Note}

\theoremstyle{plain}
\newtheorem*{theorem}{Theorem}

\newcommand{\vF}{\mathcal{F}}
\newcommand{\vG}{\mathcal{G}}
\newcommand{\vS}{\mathcal{S}}
\newcommand{\vu}{\mathbf{u}}
\newcommand{\vv}{\mathbf{v}}
\newcommand{\vw}{\mathbf{w}}


\begin{document}

\title{Introduction to Mastermind Strategies}
% Alternative book name:
%  - (Practical) Mastermind Strategies
%  - Mastering Mastermind
\author{Author}
\date{2012}
\maketitle

\begin{abstract}

This document provides a systematic review of the main strategies for playing the Mastermind game and some of its variants, including the well-studied Bulls and Cows game. Extensive literature review is conducted and a sample program is provided at the end.

Chapter 1 gives an overview of the Mastermind game and its strategies. Basic concepts and properties about the game are formalized such as a strategy tree. At the end of the chapter, the (standard) notations used to represent a strategy tree in literature is introduced.

Chapter 2 describes several heuristic strategies, including \minmax{}, \minavg{}, \maxent{} and \maxpar{}. The rationale of each heuristic as well as their performance and limitations are analyzed. At the end of the chapter lists a few minor heuristics together with a comparison of their performance under common settings of game.

Chapter 3 discusses the techniques that are commonly employed to search for an optimal strategy. Of main importance are guess equivalence detection and Alpha-Beta pruning. We introduce three types of equivalence criteria varying in degree of complexity and performance. [We (will) also discuss pruning.]

Chapter 4 provides an overview of randomized strategies aimed at solving large-scale games efficiently. The need for a randomized strategy is justified by the fact that the Mastermind Satisfiability Problem is NP-complete. [To be completed]

Chapter 5 introduces a few variants of the game and discusses the corresponding strategies. Static Mastermind is a game where all the guesses are made all at once. Dynamic Mastermind is where the code-maker can silently change the secret during the course of the game. Finally, Mastermind with a lie is where at most one of the response can be fake.

Chapter 6 details a computer program implementation of the strategies discussed above.

\end{abstract}


\setcounter{tocdepth}{1}
\tableofcontents

%[*] entries marked with an asterisk are advanced topics. Skipping these topics doesn't affect a basic understanding of the subject. However they are needed for a practical implementation of the strategies.

% Chapter 1. Introduction
%   - Overview (history, rules, variations)
%   - (Interactive) Examples
%   - Overview of strategies (introduce the strategies, and introduce a table that describes how many codewords need to be solved in how many steps)
% [secrets be revealed]
%   - A Simple Strategy
%   - Obviously-Optimal guesses
%   - Strategy tree

% Chapter 2. Heuristic Strategies
%   - Overview [How it works]
%   - The MinMax heuristic
%   - The Minavg heuristic
%   - The Max Entropy heuristic
%   - The Max Parts heuristic
%   - The Min Steps heuristic
%   - Hybrid heuristics (e.g. WideDev, LongRect)
%   - Comparison of heuristics

% Chapter 3. Optimal Strategies
%   - Overview
%   - Equivalence of guesses [?] (Filter Equivalence)
%     - Color Equivalence
%     - Constraint Equivalence
%     - Partition Equivalence
%   - Isomorphism of guesses [*] (State/Partition Equivalence)
%   - Search space pruning [*]
%   - Using a pre-built strategy tree [*]
%   - Extended/Adaptive strategy tree [*] (i.e. the tree not only contains guesses along the chosen strategy path, but also includes guesses if the user made a non-optimal guess halfway. The tree size in this case is much larger, and we must use isomorphism to detect the symmetry.)

% Chapter 4. Randomized Strategies
%   - Mastermind Satisfying Problem is NP-Complete
% [know nothing about this]

% Chapter 5. Variations (Other Topics, Related Topics)
%   - With Lies
%   - Static Mastermind
%   - Dynamic Mastermind
%   - Other improvements (e.g. two-phase optimization, hash collision group)

% Chapter 6. C++ Implementation
%   - Overview
%   - Data Structure
%     - Codeword
%     - Feedback
%     - Strategy Tree
%   - Basic operations
%     - Codeword comparison
%     - Frequency table generation
%     - Codeword set partitioning
%   - Implementing Heuristic Strategies
%     - Obvious guesses
%   - Implementing Optimal Strategies
%   - Implementing randomized strategies

% Reference

% Appendix
%   - Entropy theory
%   - Graph theory (in particular, graph isophosim)
%   - Optimial strategy listing for Mastermind 
%   - Optimal strategy listing for Guess-Number
%   - Comparison of different strategiesStrategy tree

\section{Introduction}

\subsection{Overview}

[history, rules, variations, etc]

Mastermind is a game played by two players, the \emph{code maker} and the \emph{code breaker}. At the beginning of the game, the code maker thinks of a secret codeword with $p$ pegs and $c$ colors. Then the code breaker begins making guesses.

For each guess, the code maker responds with a \emph{feedback} in the form of xAyB to describe how close the guess is to the secret. Here $x$ is the number of correct colors in the correct pegs, and $y$ is the number of correct colors in the wrong pegs. The game continues until the code breaker finds the secret.

The standard Mastermind game is played on a board with four pegs and using six colors. A codeword can contain the same color more than once. A popular variation of the rules is called GuessNumber, also known as Bulls-n-Cows. Under these rules, no repetition of colors is allowed in the codeword.

\subsection{Examples}

(Interactive) Examples

\subsection{Overview of strategies}

[introduce the strategies, and introduce a table that describes how many codewords need to be solved in how many steps). [secrets be revealed]

There are three types of strategies for the code breaker:

Simple strategy. (Ch02) The code breaker just makes a random guess, as long as the guess will bring some information. A natural choice could be the first codeword from the remaining possibilities. These strategies obviously perform poorly in that it usually takes many steps to find the secret, but they are simple and fast and thus can be used as a benchmark for other more sophisticated strategies.

Heuristic strategy. (Ch03) The code breaker evaluates each potential guess according to some scoring criteria, and picks the one that gets the highest score. These strategies are fast enough for real-time games, and performs quite well. Most research in this field focuses on finding a good heuristic criteria, and a number of intuitive and well-performing heuristics have been proposed. For more details, see heuristic strategies.

Optimal strategy. (Ch04) Since the number of possible secrets as well as sequence of guesses are finite, the code breaker can employ a depth-first search to find out the optimal guessing strategy. The optimal strategy minimizes the expected number of guesses, optionally subject to a maximum-guesses constraint. Finding such a strategy involves exhaustive search, and therefore is too slow to be suitable for real-time application. For more details on the implementation, see optimal strategies.

\subsection{A Simple Strategy}

Chooses the \emph{first} possibility. Here first depends on the order of the list. We can let it be lexicographical first. 

However, an algorithm that depends on lexicographical ordering is not very ideal. See comparision of heuristics in chapter 2 for a discussion.
 
\subsection{Strategy tree}

A standard strategy tree contains the strategy starting from the initial state.

An extended (or adaptive, full) strategy tree contains a strategy for every possible state of the game.

Introduce the notations (Irving) to aid the reader in reading the classical papers.
  
\subsection{Lower-bound of the number of guesses}

% -------------------------------- %
























% -------------------------------- %
\chapter{Equivalence of Guesses}

\section{Overview}

Just as a human would be easy to identify equivalent guesses, we use that in computer as well.

[Isomorphism of guesses]

\section{Color equivalence}

As a start-up, let's work with a simple but useful way to detect guess equivalence.

\subsection{Definitions}

\begin{definition}
(Color mask) A \emph{color mask}, denoted $\tau$, is a subset of the available colors. Since the number of colors in a game is small (6 for Mastermind and 10 for Bulls and cows), it is convenient to implement this subset as a bit-mask with present colors set to one. That's why we call it a mask here.
\end{definition}

Given a codeword $g$, define $\tau(g)$ to be the colors present in $g$. For example, $\tau(\cw{1442}) = \{1, 2, 4\}$. 

Given a codeword set $\mathcal{S}$, define $\tau(\mathcal{S})$ to be the colors present in any of the codewords in $\mathcal{S}$. That is, $\tau(\mathcal{S}) = \bigcup_{g \in \mathcal{S}} \tau(g)$. For example, $\tau(\{\cw{1442},\cw{2315} \}) = \{ 1, 2, 3, 4, 5\}$.

of a codeword is the set of colors present in the codeword. The color mask of a set of codewords is the set of colors present in any of the codewords.

A few of the important color masks are as follows.

$\tau_{\text{all}}$ is the set of all colors.

$\tau_{\text{free}}$ is the set of colors that have never been guessed. $\tau_{\text{used}}$ is the set of colors that have been guessed. We have $\tau_{\text{free}} \cup \tau_{\text{used}} = \tau_\text{all}$.

$\tau_{\text{excl}}$ is the set of colors that are excluded from the potential secrets.


\subsection{Algorithm}

\newcommand{\cmall}{\tau_\text{all}}
\newcommand{\cmfree}{\tau_\text{free}}
\newcommand{\cmexcl}{\tau_\text{excl}}

Denote by $\mathcal{F}_k$ a color equivalence filter. It is characterized by a pair of color masks: the free/fresh/unguessed/uncalled colors and the excluded colors. That is,
\[
\mathcal{F} = (\tau_\text{free}, \tau_\text{excl}) .
\]

Three operations are:

\emph{Initialize}. At the beginning of the game, there are no constraints. A color equivalence filter is initialized by
\[
\cmfree = \cmall, \cmexcl = \phi .
\]

\emph{Filter}. For each candidate $g \in \mathcal{G}$, check whether $g$ is the lexical minimum of its equivalence class. We check this by replacing all excluded colors with the same symbol (e.g. *), and replace the unguessed colors using suitable mapping.

\emph{Restrict}. Given a constraint $(g,r)$ and the set of remaining possibilities $\mathcal{S}$, we construct a new filter as $\mathcal{F}' = (\cmfree', \cmexcl')$, where
\[
\cmfree' = \cmfree \setminus \tau(g), \cmexcl' = \cmall \setminus \tau(\mathcal{S})  .
\]


\section{Constraint equivalence}

Here the definition of \emph{constraint equivalence} is rather restrictive. Ideally, constraint equivalence should be the same as partition equivalence, but this is not the case here.

Introduced by \cite{neuwirth81,koyama93}. 

[Constraint/filter equivalence]

It is easy to see that solving a game only depends on what colors have appeared before, and the relative peg position of the guesses. The particular colors or pegs guessed are not important. From this knowledge, we define the following.

\subsection{Definitions}

\begin{definition}
(Peg permutation) A \emph{peg permutation}, denoted $\pi_p$, permutes the pegs in a codeword. For example, the peg permutation
\[
\begin{pmatrix}
1 & 2 & 3 & 4 \\
3 & 1 & 2 & 4
\end{pmatrix} 
\]
moves peg 1 to the 3rd place, peg 2 to the 1st place, peg 3 to the 2nd place, and peg 4 in its original place. The reordered sequence of the pegs is given by $\pi_p^{-1}$, which in this example is equal to
\[
\begin{pmatrix}
1 & 2 & 3 & 4 \\
2 & 3 & 1 & 4
\end{pmatrix} .
\]
\end{definition}

\begin{definition}
(Color permutation) A \emph{color permutation}, denoted $\pi_c$, permutes the colors in a codeword. For example, the color permutation
\[
\begin{pmatrix}
1 & 2 & 3 & 4 & 5 & 6 \\
5 & 2 & 1 & 6 & 3 & 4
\end{pmatrix} ,
\]
replaces color 1 by color 5, color 2 as is, color 3 by color 1, etc.
\end{definition}

\begin{definition}
(Partial color permutation) A \emph{partial color permutation} permutes a subset of colors. The set of colors not involved in the partial color permutation are called \emph{free colors} and denoted by $\tau$.
% , and leave the rest colors unchanged
%with the additional attribute that only a subset of the mappings are subject to future changes. These mappings are called \emph{free} mappings. 
%, denoted $\Pi_c$, is an incomplete permutation of the colors where the mapping among some colors are fixed, but the mapping among the rest colors are not specified. 
For example, the partial color permutation
\[
\begin{pmatrix}
%1^* & 2^* & 3^* & 4^* & 5^* & 6^*  \\
%1^* & 2^* & 3^* & 4^* & 5^* & 6^* 
%1 & 2 & 3 & 4 & 5 & 6 \\
%\textbf{1} & 2 & 3 & 4 & 5 & 6 \\
%* & * & * & * & * & *
\underline{1} & \underline{2} & 3 & \underline{4} & 5 & \underline{6} \\
\underline{1} & \underline{2} & 5 & \underline{4} & 3 & \underline{6} 
\end{pmatrix} .
\]
permutes (and \emph{restricts}) colors \cw{3} and \cw{5}, and leaves the free colors (marked with an underline) unchanged.
\end{definition}

A class of (fully-restricted) color permutations can be generated from a partial color permutation $\pi_c$ and its associated set of free colors $\tau$ by composing the partial permutation with every possible permutation of the free colors. We denote such class of color permutations by $\Pi_c = \pi_c \circ S_\tau$.

%A partial permutation in this context is actually a \emph{permutation group} [see appendix] formed by composing the fixed mappings with all permutations of the free colors. We can conveniently denote it as $\Pi_c = \pi_c \circ \tau$, where $\pi_c$ is the color mapping already fixed, and $\tau$ is the set of unmapped colors.

\begin{definition}
(Codeword permutation) A \emph{codeword permutation}, denoted $\pi$, is the combination of a peg permutation $\pi_p$ and a (partial) color permutation $\pi_c$. We write it as $\pi = \pi_p \circ \pi_c$. Note that the actual order of applying the peg permutation or the color permutation first does not matter.
\end{definition}

Given codeword $g$, we denote the permuted codeword under $\pi$ as $\pi(g)$ or $g^\pi$. 
%Note that since a codeword usually does not contain all available colors, the mapping of unused colors does not matter when applying a color permutation to a given codeword.
To compute the image $h = (h_i)$ of a given codeword $g = (g_i)$ under a given permutation $\pi = \pi_p \circ \pi_c$, use the formula
\[
h_i = \pi_c\left(g_{\pi_p^{-1}(i)}\right) .
\]

%[Properties.] Some usual properties of permutations apply to codeword permutation as well.
%
%1) A codeword permutation is invertible, i.e.\ for any $\pi = (\pi_p, \pi_c)$, its inverse exists and is equal to $\pi^{-1} = (\pi_p^{-1}, \pi_c^{-1})$.

\begin{definition}
(Codeword equivalence) Two codewords $g$ and $h$ are \emph{equivalent} if and only if there exists a codeword permutation $\pi$ such that $g^\pi = h$.
\end{definition}

The equivalence relation defined above partitions a set of codewords into equivalence classes, where each equivalence class can be represented by an arbitrary element in that class, called a \emph{representative}. 
%All elements in the same equivalence class are equivalent to this representative.
%; i.e.\ for all $g$, there exists $\pi$ such that $g = g_0^\pi$.

\begin{definition}
(Constraint equivalence) [cite a few references] Let $\mathcal{C}_1$ and $\mathcal{C}_2$ be two ordered sets of constraints of the same size $k$. Let the guesses in $\mathcal{C}_1$ be $(g_1,g_2,\cdots,g_k)$ and the guesses in $\mathcal{C}_2$ be $(h_1,h_2,\cdots,h_k)$. Then constraint sets $\mathcal{C}_1$ and $\mathcal{C}_2$ are \emph{equivalent} if there exists a codeword permutation $\pi$ such that
$g_i^\pi = h_i$ for all $1 \le i \le k$.
\end{definition}

The intuition about equivalent constraints are rather simple: if we rearrange the peg positions and relabel the colors, then the two sets of constraints become exactly the same. Hence for the purpose of devising a strategy, we only need to work with one of them and the other follows automatically.

Note, however, that there is a strong restriction in the above definition: the responses in the constraints are ignored. This has two implications. On the one hand, constraints such as 
\[
\mathcal{C}_1 = \{ (1234:0A0B), (3456:2A0B) \}
\]
and
\[
\mathcal{C}_2 = \{ (1234:1A0B), (3456:0A2B) \}
\]
are considered equivalent, although they lead to drastically different potential secrets. On the other hand, constraints such as
\[
\mathcal{C}_1 = \{ (1234:0A0B), (1234:0A0B) \}
\]
and
\[
\mathcal{C}_2 = \{ (1234:0A0B), (4321:0A0B) \}
\]
are obviously not equivalent by definition, but from the first feedback \fb{0}{0} we could already exclude the colors \cw{1} -- \cw{4} from the potential secret, so any permutation of \cw{1234} should ideally be considered equivalent.

Despite the less-than-ideal properties of such definition, this equivalence relation is still commonly used \cite{neuwirth81,koyama93,francis10} because of its clarity, relative simplicity, and effectiveness for the first few guesses. A more powerful (but also more complex) equivalence definition will be introduced in the next section.

\subsection{Illustration}

To actually apply a constraint equivalence filter in practice, we implement an \emph{incremental filter}, as illustrated below.\footnote{The incremental filter described in this section is effectively the same technique used by \cite{francis10} to detect equivalence in the Bulls and cows game.}
An alternative, more generic method, which relies on \emph{graph automorphism}, will be introduced in the next section.

Given a set $\mathcal{C}$ of constraints and a set $\mathcal{G}$ of candidate codewords to filter for the next guess, we want to find all canonical codewords $g \in \mathcal{G}$ to be used as the next guess. There are a couple of ways to do this,\footnote{For example, one could \emph{generate} a list of canonical guesses from scratch, or \emph{filter} a supplied list of codewords for canonical guesses. Here we employ the second approach because of its simplicity, flexibility, and good performance in an implementation.}
and we employ a simple algorithm: traverse the set and keep the codewords that are lexically minimal in its equivalence class. That is, for each $g \in \mathcal{G}$, we find the lexically minimum codeword $g_0$ that is equivalent to $g$, and keep $g$ if and only if $g = g_0$. 
%We call $g_0$ the \emph{canonical representative} of $g$.

Note that in order for the algorithm to work correctly, the lexically minimal codeword of each equivalence class must be present in the candidate set $\mathcal{G}$.\footnote{
This requirement is satisfied in most practical application. The algorithm could be slightly modified to remove this requirement: instead of removing a non-canonical element, it could employ a subset union algorithm to label the equivalence class of each element. After all candidates are processed, it can scan the labels to find the canonical elements.
}


To find the lexically minimum codeword equivalent to $g$, we could permute $g$ in all possible ways and check if any of the permuted image is smaller than $g$. For a Mastermind game, there are 4 pegs and 6 colors. The total number of permutations is therefore $4! \times 6! = 24 \times 720 = 17280$. This number is much larger in the Bulls and Cows game with 4 pegs and 10 colors, which is equal to $4! \times 10! = 87,091,200$.

Therefore we need to iterate the permutations in a more efficient way. We first conveniently classify the codeword permutations by their peg permutation, i.e.\ those with the same peg permutation are put into the same class. Then, as it will turn out shortly, the permutations in each class can be written as
\[
\pi_p \circ \pi_c \circ S_\tau ,
\]
where $\pi_p$ is the peg permutation that represents the class, $\pi_c$ is a partial color permutation restricted by the supplied constraints under this peg permutation, and $S_\tau$ is the group of all permutations of the set of free colors.

%We first find the canonical representative given each possible peg permutation of $g$. We then take the minimum of these minimum as the global minimum. [TBC]

Let's illustrate the algorithm with 4 pegs and 6 colors. We first list all $4! = 24$ permutations of the pegs, $\pi_p^1$ to $\pi_p^{24}$, where $\pi_p^1$ is the identity permutation.

Next, for each peg permutation $\pi_p^i$, we associate with it a class of \emph{eligible} color permutations, $\Pi_c^i = \pi_c^i \circ S_\tau$, where $\pi_c^i$ is the partial color permutation restricted by the supplied constraints under this peg permutation, and $\tau$ is the set of free colors. 
%$\Pi_c$ hence denotes the set of all eligibal color permutations as the product of $\pi_c^i$ and all permutations of the colors in $\tau$. 

At the beginning of the game, there are no constraints; all colors are free, and every partial color permutation is unrestricted. We can conveniently write $\pi_c \circ S_\tau$ as
%We can conveniently write an identity permutation in the set of eligible color permutations as
\[
%\Pi_c^i = 
\begin{pmatrix}
%1^* & 2^* & 3^* & 4^* & 5^* & 6^*  \\
%1^* & 2^* & 3^* & 4^* & 5^* & 6^* 
%1 & 2 & 3 & 4 & 5 & 6 \\
%\textbf{1} & 2 & 3 & 4 & 5 & 6 \\
%* & * & * & * & * & *
\underline{1} & \underline{2} & \underline{3} & \underline{4} & \underline{5} & \underline{6} \\
\underline{1} & \underline{2} & \underline{3} & \underline{4} & \underline{5} & \underline{6} 
\end{pmatrix} ,
\]
where the underlined colors are controlled by $S_\tau$ and hence can be mapped freely.

To filter a set of codewords $\mathcal{G}$ to obtain canonical guesses, we check each codeword $g \in \mathcal{G}$ in order.\footnote{The particular order of traversal is not important, as long as the lexical minimum belongs to the set.} 
Take for example $g = \cw{1223}$. We iterate through each peg permutation $\pi_p^i$ to permute $g$. Apply for example the following peg permutation,
\[
\pi_p = 
\begin{pmatrix}
1 & 2 & 3 & 4 \\
2 & 1 & 3 & 4
\end{pmatrix} ,
%\]
%whose associated partial color permutation is initially
%\[
%\Pi_c = 
%\begin{pmatrix}
%1^* & 2^* & 3^* & 4^* & 5^* & 6^*  \\
%1^* & 2^* & 3^* & 4^* & 5^* & 6^* 
%\end{pmatrix} .
\]
followed by the (eligible) identity color permutation.
The permuted codeword is $g' = \cw{2123}$. Since $g' \succ g$, we cannot yet conclude that $g$ is not lexically minimum in its equivalence class. We proceed as follows.

We are free to map the free colors in $g'$ to any free color. Given our objective is to find the lexical minimum, we start from the leftmost peg of $g'$, which contains the color \cw{2}. This is a free color, and it should be mapped to the smallest available free color, \cw{1}, to achieve lexical minimum. This also means we have to map the color \cw{1} to something else temporarily because we cannot have two colors map to the same value. For convenience we map it to \cw{2}. Thus we apply the following color permutation to $g' = \cw{2123}$,
\[
\pi_c = 
\begin{pmatrix}
\underline{1} & 2 & \underline{3} & \underline{4} & \underline{5} & \underline{6} \\
\underline{2} & 1 & \underline{3} & \underline{4} & \underline{5} & \underline{6} 
\end{pmatrix} ,
\]
which yields $\pi_c(g') = \cw{1213}$.

Note that \cw{1213} is already lexically smaller than \cw{1223}, so we can conclude that \cw{1223} is not the lexical minimum of its equivalence class and can thus remove it from the candidates. However, for illustration purpose here we show a few more steps.

Now proceed to the color on the second peg of $g'$, \cw{1}. Again, \cw{1} is not mapped in the color permutation, so we are free to choose its image. And for the same reason of achieving minimal lexicographical value, we map it to the smallest available free color, which in this case is \cw{2}. The resulting color permutation is
\[
\pi_c = 
\begin{pmatrix}
1 & 2 & \underline{3} & \underline{4} & \underline{5} & \underline{6} \\
2 & 1 & \underline{3} & \underline{4} & \underline{5} & \underline{6} 
\end{pmatrix} ,
\]
and the permuted codeword is $\pi_c(g') = \cw{1213}$. Repeat this until all pegs in $g'$ are processed, and we find the lexical minimum to be \cw{1213}, which is smaller than $g$. We then remove $g$ from the candidate set. 

Repeat the above steps for all peg permutations. If $g$ is lexically minimum in all these cases, then it is the lexical minimum of its equivalence class, and we can add it to the filtered set.

Above is an examine where all colors are free. Now suppose we have already have some constraints, so some colors are fixed. Without loss of generality, suppose that the constraint consists of just one guess $g_1 = \cw{3445}$. To find all canonical candidates for the second guess, we first need to \emph{restrict} the partial color permutations associated with each peg permutation. 

Take for example 
\[
\pi_p = 
\begin{pmatrix}
1 & 2 & 3 & 4 \\
1 & 3 & 2 & 4
\end{pmatrix} , 
\Pi_c = 
\begin{pmatrix}
\underline{1} & \underline{2} & \underline{3} & \underline{4} & \underline{5} & \underline{6} \\
\underline{1} & \underline{2} & \underline{3} & \underline{4} & \underline{5} & \underline{6} 
\end{pmatrix} .
\] 
The permuted first guess is $\pi(g_1) = \cw{3445}$. In order for the permuted constraint to be equivalent, the permuted guess must be equal to $g_1$. This means we must map \cw{3445} to \cw{3445} in the color permutation. Thus we must restrict the associated color permutation as
\[
\Pi_c' = 
\begin{pmatrix}
\underline{1} & \underline{2} & 3 & 4 & 5 & \underline{6} \\
\underline{1} & \underline{2} & 3 & 4 & 5  & \underline{6} 
\end{pmatrix} .
\]

After restricting the partial permutation, we can apply the previous steps again. 
[The following example need to be changed to continue from the previous example] Suppose we come again to \cw{1123}. If we apply peg permutation $\pi_p^3$, the permuted representative becomes \cw{1213}. Now to find out what color permutations are eligible, note that the color 3 is already fixed in the partial permutation; only colors 1 and 2 are free to be mapped. We could therefore map them freely among the unrestricted colors, i.e.\ $\{1, 2, 6\}$, and mark all resulting codewords as equivalent to \cw{1123}.

Note that given a set of constraints, not all peg permutations can yield an eligible color permutation. For example, consider the peg permutation
\[
\pi_p = 
\begin{pmatrix}
1 & 2 & 3 & 4 \\
1 & 2 & 4 & 3
\end{pmatrix} .
\]
When applied to $g_1 = \cw{3445}$, we get $\pi_p(g_1) = \cw{3454}$. However, there is no way whatsoever to map \cw{3445} to \cw{3454}, because \cw{4} would have to be mapped to both \cw{4} and \cw{5}, which is impossible. In this case, we conclude that this peg permutation is \emph{ineligible} and remove it from further consideration.

\subsection{Algorithm}

%Given a set of $k$ constraints, 
%%$\mathcal{C}_k = \{ (g_i,r_i) \given 1 \le i \le k \}$, 
%$\mathcal{C}_k = \{ (g_1,r_1), \cdots, (g_k, r_k) \}$, 
%a \emph{constraint equivalence filter}, denoted $\mathcal{F}_k$, produces a short-list of canonical codewords for the next guess.

A constraint equivalence filter is characterized by the set of eligible codeword permutations which map existing constraints to themselves. These permutations 
can be classified into $m$ classes according to their peg permutation. The permutations in the $i$th class can be written as the product of the peg permutation $\pi_p^i$, the associated partial color permutation $\pi_c^i$, and the symmetric group of the free colors $S_\tau$ (which is the same across all $i$). 

Thus, we can write a constraint equivalence filter as
\[
\mathcal{F}_k = \bigcup_{i=1}^m \pi_p^i \circ \pi_c^i \circ S_\tau ,
\]
where $m$ is the number of eligible peg permutations.

Three operations are defined for a constraint equivalence filter:

\emph{Initialize}. At the beginning of the game, there are no constraints. Hence the initial filter, $\mathcal{F}_0$, is set to 
\[
m = \text{number of pegs}!, \pi_c^i = (), \tau = \{1, 2, \ldots, c \}.
\]

\emph{Filter}. To filter a set $\mathcal{G}$ of candidate codewords for canonical guesses, we check each codeword $g \in \mathcal{G}$ in order.\footnote{The particular order of traversal is not important, as long as the lexical minimum belongs to the set.} 
Given $g$, we check each eligible peg permutation in $\mathcal{F}$. For peg permutation $i$, we first permute $g$ to get $g' = \pi_p \circ \pi_c (g)$. If $g' \prec g$, then $g$ is not minimum. Otherwise, let $g' = (c_1, c_2, c_3, c_4)$. For each $j = 1$ to $4$, if $j \in \tau$, then map $j$ to the smallest color in $\tau$, etc.


%Formally, let $\mathcal{F} = \left\{(\pi_p^i, \pi_c^i \circ \tau) \given 1 \le i \le m \right\}$ be an incremental equivalence filter containing $m$ (partial) codeword permutations. Every partial color permutation has the property that some of the colors are restricted, while the rest are not. Let the set of free colors be $\tau_i$. As will show shortly, $\tau_i$ is the same across all permutations in a filter, so we can denote them by $\tau$. Also, for each $\Pi_c^i$, we pick a representitive $\pi_c^i \in \Pi_c^i$. Hence the filter can be written as
%\[
%\mathcal{F} = \left\{(\pi_p^i, \pi_c^i; \tau) \given 1 \le i \le m \right\} .
%\]

\emph{Restrict}. Given a filter $\mathcal{F}_k$ that satisfies all existing constraints $\mathcal{C}_k$. When a constraint $(g, r)$ is added, we construct a new filter $\mathcal{F}_{k+1} \subseteq \mathcal{F}_k$ by selecting the permutations in $\mathcal{F}_k$ that maps $g$ to itself. This is effectively restricting the partial color permutation associated with each peg permutation.
%so that
%$\pi_c^i ( \pi_p^i (g) ) = \pi_p^i (g)$ for all $\pi_c^i \in \Pi_c^i$. 
%$\pi(g_k) = g_k$ for all $\pi \in \mathcal{F}_{k+1}$.

[TBC] It can be seen from the above requirement that any color that is mapped from in a partial color partition is also mapped to, and any color that is not mapped in one direction is also not mapped in the other direction. In fact, the unmapped colors are those \emph{unused} in any of the prior constraints. This means this equivalence filter fully considers the ``unguessed color equivalence'' described in the previous section.


As we proceed in the game, we are supplied with more constraints. The partial color permutation associated with each peg permutation gets more restrictive with each
added constraint, and more peg permutations becomes ineligible and gets removed. Finally we will be left with only the identity codeword permutation, where every codeword will become a representative of its own equivalence class.
%Hence we call it "incremental equivalence detection".

\subsection{Results}

Applying the constraint equivalence filter described in this section, we find 5 canonical guesses for the first round of a standard Mastermind game (p4c6r), listed in Table \ref{tab:canonical-mastermind}. Listed alongside is the number of canonical guesses in the second round given each initial guess.
\begin{table}[h]
\begin{center}
\begin{tabular}{c c}
\hline
\hline
$g_1$ & $\#\{g_2\}$ \\
\hline
\cw{0000} & 12 \\
\cw{0001} & 53 \\
\cw{0011} & 39 \\
\cw{0012} & 130 \\
\cw{0123} & 57 \\
\hline
\hline
\end{tabular}
\caption{Canonical 1st and 2nd guesses in Mastermind}
\label{tab:canonical-mastermind}
\end{center}
\end{table}

Applying the same filter to the Bulls and cows game (p4c10n) yields only one canonical initial guess, \cw{0123}, and 20 canonical guesses for the second round. They are listed in Table \ref{tab:canonical-bulls} along with the number of canonical guesses for the third round.\footnote{The table on page 7 of \cite{francis10} contains the same information; however their number for \cw{0456} is 373 and their number for \cw{4567} is 218.}
\begin{table}[h]
\begin{center}
\begin{tabular}{c c | c c | c c | c c}
\hline
\hline
$g_2$ & $\#\{g_3\}$ & $g_2$ & $\#\{g_3\}$ & $g_2$ & $\#\{g_3\}$ & $g_2$ & $\#\{g_3\}$ \\
\hline
\cw{0123} & 20  & \cw{0214} & 270  & \cw{1032} & 39  & \cw{1234} & 501  \\
\cw{0124} & 107 & \cw{0231} & 75   & \cw{1034} & 270 & \cw{1245} & 1045 \\
\cw{0132} & 67  & \cw{0234} & 501  & \cw{1045} & 295 & \cw{1435} & 541  \\
\cw{0134} & 270 & \cw{0245} & 1045 & \cw{1204} & 175 & \cw{1456} & 1012 \\
\cw{0145} & 295 & \cw{0456} & 363  & \cw{1230} & 59  & \cw{4567} & 180  \\
\hline
\hline
\end{tabular}
\caption{Canonical 2nd and 3rd guesses in Bulls and cows}
\label{tab:canonical-bulls}
\end{center}
\end{table}

[consistent with all prior guesses]

\section{Partition equivalence}

State/Partition Equivalence

This is the finest-grained definition of equivalence, but is complex in implementation.

\begin{definition}
(Partition) Let $(b_1, b_2, \cdots, b_r)$ be the ordered set of distinct feedbacks in a game. Then the \emph{partition} of a codeword set $\mathcal{S}$ by a codeword $g$ is a partition of $\mathcal{S}$ into ordered cells $(V_1, V_2, \cdots, V_r)$ such that the codewords in $V_i$ compare to $g$ yields $b_i$.
\end{definition}

\begin{definition}
(Partition equivalence.) Given two partitions $P_1 = (V_1, V_2, \cdots, V_r)$ and $P_2 = (V'_1, V'_2, \cdots, V'_r)$ of a codeword set $\mathcal{S}$. $P_1$ and $P_2$ are \emph{equivalent} if there exists codeword permutation $\pi$ such that $V_i^\pi = V'_i$ for all $1 \le i \le r$. That is, $\pi$ maps the cells in $P_1$ to the same cells in $P_2$. [color preserving permutation]
\end{definition}

\begin{definition}
(Guess equivalence.) Given a set of potential secrets $\mathcal{S}$, two guesses $g_1$ and $g_2$ are \emph{equivalent} if the partitions given by $g_1$ and $g_2$ on $\mathcal{S}$ are equivalent.
\end{definition}

[Can we show that partition equivalence is the \emph{finest} equivalence relation that preserves the \emph{structure} of a strategy tree?] Of course we need to define the motivation and what is a ``structure'' first.

\section{Chaining Multiple Filters}

The equivalence filters described above vary in complexity and performance. More complex filters tend to produce fewer canonical guesses, but at the cost of higher computational overhead. If the computational overhead is too high, the benefit of filtering could be offset.

It is therefore useful in practice to chain multiple equivalence filters together by feeding the output of one filter as input into the next filter. This has two benefits: i) it potentially reduces the number of canonical guesses by testing different equivalence relations, and ii) it speeds up the computation of the latter filter by supplying potentially fewer candidates as input.\footnote{This relies on the assumption that the computational complexity of a filter grows with the number of candidates, which is satisfied for the filters described in the previous sections.}
Below we discuss some issues in implementing such a filter chain.

\begin{definition}
(Composite equivalence) Let $R_1$ and $R_2$ be two equivalence relations defined on the codeword set $\vS_0$. Then $R$ is a \emph{composite equivalence relation} of $R_1$ and $R_2$ if for any $g_1, g_2 \in \vS_0$, $g_1 R g_2$ if and only if $g_1 R_1 g_2$ or $g_1 R_2 g_2$. We denote it as $R = R_1 \vee R_2$. The $\vee$ operator is commonly known as the \emph{join} of the partitions induced by the two equivalence relations.\footnote{See \url{http://en.wikipedia.org/wiki/Lattice\_(order)}.}
\end{definition}

\begin{definition}
(Equivalence refinement) For two equivalence relations $R_1$ and $R_2$, $R_1$ is called \emph{finer} than $R_2$ (and $R_2$ \emph{coarser} than $R_1$) if any pair of elements that are equivalent under $R_1$ are also equivalent under $R_2$. It is easy to see that if $R_1$ is finer than $R_2$, then $R_1 \vee R_2 = R_2$.
\end{definition}

\begin{definition}
(Compatible relations) Let $R_1$ and $R_2$ be two equivalence relations defined on the set $S$. Then $R_1$ and $R_2$ are called \emph{compatible} if for any $g \in \vS$, $[g]_{R_1} \subseteq [g]_{R_2}$ or $[g]_{R_2} \subseteq [g]_{R_1}$. Intuitively, this means $R_1$ is a refinement or $R_2$ for part of the set, and $R_2$ is a refinement of $R_1$ for the rest part of the set. In particular, any equivalence relation is compatible with its refinement.
\end{definition}

\begin{definition}
(Discrete equivalence) A \emph{discrete} equivalence relation is an equivalence relation where every element is equivalent only to itself. A discrete equivalence relation is finer than any other equivalence relation defined on the same set.
\end{definition}

\begin{definition}
(Unit equivalence) A \emph{unit} equivalence relation is an equivalence relation where all elements are equivalent. A unit equivalence relation is coarser than any other equivalence relation defined on the same set.
\end{definition}

\begin{definition}
(Trivial equivalence) Discrete and unit equivalence relations are collectively called \emph{trivial} equivalence relations.
\end{definition}

\begin{definition}
(Simple filter) Let $R$ be an equivalence relation defined on a set $S$. A \emph{simple filter} of $\vS$ with respect to $R$ is a function that takes as input a subset $\vG \subseteq \vS$ and produces as output a set containing a representative for each class of equivalent elements in $\vG$.\footnote{Such output is known as the set of \emph{class representatives} of $\vG$; see \url{http://mathworld.wolfram.com/ClassRepresentative.html}. }
\end{definition}

One way to implement a composite equivalence filter is using \emph{disjoint sets}.\footnote{See \url{http://en.wikipedia.org/wiki/Disjoint-set\_data\_structure}.} However, this approach does not reduce the computational complexity of the second filter, and is not useful when one equivalence is finer than the other. In addition, the data structure is more complex to implement.

Therefore, we prefer to implement a composite filter by \emph{chaining} the individual filters. That is, given candidate set $\vG$, we first apply the first filter to get $\vG_1 = \vF_{R_1}(\vG)$, and then apply the second filter to get $\vG_2 = \vF_{R_2}(\vG_1)$. If we can ensure $\vG_2 = \vF_{R_1 \vee R_2}(\vG)$, then this method is a simple and efficient way to compute the composite.

However, arbitrary equivalence filters for arbitrary equivalence relations \emph{cannot} be chained in the above manner to produce the correct result. Let's look at a counter-example. Take a set of three elements, $X = \{1, 2, 3\}$, and take the following equivalence relations:
\begin{align}
R_1 &= \{ \{1, 2\}, \{3\}  \} , \notag \\
R_2 &= \{ \{1\}, \{2,3\}  \} . \notag 
\end{align}
It follows that the composite equivalence relation is $R_1 \vee R_2 = \{ \{ 1, 2, 3 \} \}$.

Now suppose we use a ``natural'' filter $\vF$ that keeps elements in the input sequence that are not equivalent to any of the prior elements. Then we have $\vF_{R_1}(X) = \{ 1, 3 \}$. If we in turn feed this output to the next filter, we get $\vF_{R_2}(\{1,3\}) = \{1,3 \}$. However, this result is not desired because $\vF_{R_1 \vee R_2}(X) = \{1\}$.

This counter-example can be generalized to show that for arbitrary choices of equivalence relations and simple filters, we are not guaranteed to get the desired result by simply chaining individual filters. This is formalized by the following theorem.

\begin{theorem}
Let $R_1$ be a non-trivial equivalence relation defined on a set $\vS$, and let $\vF_{R_1}$ be a simple filter. Then there exists equivalence relation $R_2$ and set $\vG \subseteq \vS$ such that for any simple filter $\vF_{R_1 \vee R_2}$ and $\vF_{R_2}$, $\vF_{R_1 \vee R_2}(\vG) \ne \vF_{R_2}(\vF_{R_1}(\vG))$.
\end{theorem}

\begin{proof}
Since $R_1$ is non-trivial, there must exist a set of three distinct elements $\vG = \{ g_1, g_2, g_3 \} \subseteq \vS$ such that $R_1$ partitions $\vG$ as
\[
\vG / R_1 = \{ \{ g_1, g_2 \}, \{ g_3 \} \} .
\]
It follows that $\vF_{R_1} (\vG) = \{ g, g_3 \}$ where $g$ is one of $g_1$ and $g_2$. 
Now define equivalence relation $R_2$ on $\vS$ as
\[
\vS / R_2 = \{ \{g\}, \vS \setminus \{g\} \} .
\]
It is easy to see that $R_1 \vee R_2$ is the unit equivalence $\{ \vS \}$, so $\vF_{R_1 \vee R_2}(\vG)$ contains only one element. However, $\vF_{R_2}(\vF_{R_1}(\vG)) = \vF_{R_2}(\{ g, g_3 \}) = \{ g, g_3 \}$.
\end{proof}

The above theorem implies that for a non-trivial equivalence relation $R_1$, it is impossible to implement a ``universal'' simple filter that can be chained to an arbitrary equivalence relation $R_2$. However, for specific choices of $R_2$, in particular those compatible with $R_1$, this actually is possible. This is shown by the following theorem.

\begin{theorem}
Let $R_1$ and $R_2$ be two compatible equivalence relations defined on a set $\vS$, and let $\vF_{R_1}$, $\vF_{R_2}$, and $\vF_{R_1 \vee R_2}$ be simple filters. Then $\vF_{R_1 \vee R_2}(\vG) = \vF_{R_2}(\vF_{R_1}(\vG))$ for all $\vG \subseteq \vS$.
\end{theorem}

This is quite obvious, so we are not going to elaborate the proof here.

\begin{theorem}
Let $R_1$ and $R_2$ be two non-compatible equivalence relations defined on a set $\vS$, and let $\vF_R$ denote a simple filter for $R$. Then for any $\vF_{R_2}$, there exists $\vF_{R_1}$ and $\vG \subseteq \vS$ such that $\vF_{R_2}(\vF_{R_1}(\vG)) \ne \vG / (R_1 \vee R_2)$.
\end{theorem}

\begin{proof}
Since $R_1$ and $R_2$ are not compatible, there must exist a set of three distinct elements $\vG = \{ g_1, g_2, g_3 \} \subseteq \vS$ such that $R_1$ and $R_2$ partitions $\vG$ as
\[
\begin{array}{ r l }
R_1: & \{ g_1, g_2 \}, \{ g_3 \} , \\
R_2: & \{ g_1 \}, \{ g_2, g_3 \} .
\end{array}
\]
It is easy to see that $g_1, g_2, g_3$ are all equivalent under $R_1 \vee R_2$, so $\vG / (R_1 \vee R_2)$ contains only one element.
Now define $\vF_{R_1} (\vG) = \{ g_1, g_3 \}$. It then follows that
\[
\vF_{R_2}(\vF_{R_1}(\vG)) = \vF_{R_2}(\{ g_1, g_3 \}) = \{ g_1, g_3 \} \ne \vG / (R_1 \vee R_2) .
\]
\end{proof}

%The above theorem shows that natural filters cannot be chained for arbitrary equivalence relations and arbitrary input. However, for specific pairs of equivalence relations, they may still be chained. In particular, if one equivalence is finer than the other, then it is easy to see that they can be chained.\footnote{This is why we require $R_1$ to be non-trivial in the above theorem, since a discrete relation is finer than any relation, and any relation is finer than a unit relation.}

The counter-example and the theorem above means in order to chain filters on arbitrary relations, we must impose additional restriction on how a filter returns representative elements. For practical usefulness, we define a class of \emph{canonical} filters which satisfy this.

\begin{definition}
(Canonical filter) A \emph{canonical filter} is an equivalence filter that only returns representatives that are minimum in their respective equivalence classes. Formally, let $\prec$ be a total order on the set $\vS_0$ of all codewords.\footnote{A convenient example of such total order is the lexicographical order.}
Let $R$ be an equivalence relation that partitions $\vS_0$ into $r$ cells, where the minimum element (with respect to $\prec$) of the $i$th cell is denoted by $m_i$. Then a \emph{canonical filter}, $\vF$ is a function $\vS_0 \rightarrow \vS_0$ defined by
\begin{equation}
\vF(\vG) = \vG \cap \left\{ m_i \given 1 \le i \le r \right\} \text{ for any } \vG \subseteq \vS_0 . \label{eq:canonical-filter-1}
\end{equation}
In particular, if some $m_i$ is not in $\vG$, then its equivalence class is excluded from the output.
\end{definition}

Next we show that we can safely chain canonical filters together to produce the desired output efficiently.

\begin{theorem}
Let $\mathcal{F}_1$ and $\mathcal{F}_2$ be the canonical filters for equivalence relations $R_1$ and $R_2$. Let $\vF$ be the canonical filter for equivalence relation $R = R_1 \vee R_2$. Then $\vF(\vG) = \vF_2(\vF_1(\vG))$ for any $\vG \subseteq \vS_0$.
\end{theorem}

\begin{proof}
Before going into details, we illustrate the idea with an example of 7 codewords. A $\times$ sign indicates a representative of one of the canonical filters, and a $\otimes$ sign indicates a representative of both filters.
\[
\begin{array}{r c c c c c c c }
      & g_1 & g_2 & g_3 & g_4 & g_5 & g_6 & g_7 \\
\vF_1: & \otimes &        & \times & \otimes & & \times & \\
\vF_2: & \otimes & \times &        & \otimes & &        & \times
\end{array}
\]

Formally, let $\vu = (u_1, \cdots, u_r)$ and $\vv = (v_1, \cdots, v_s)$ be the ordered set of minimum representatives of $\vS_0 / R_1$ and $\vS_0 / R_2$ respectively. According to equation \eqref{eq:canonical-filter-1}, we have
\[
\vF_1(\vG) = \vG \cap \vu, \vF_2(\vG) = \vG \cap \vv.
\]
Let $\vw = \vu \cap \vv = (w_1, \cdots, w_t)$. We now show that $\vw$ is the set of minimum representatives of $\vS_0 / (R_1 \vee R_2)$.

First, it is obvious that any $w_{k_1}, w_{k_2} \in \vw$ are not equivalent under $R_1$ or $R_2$. Otherwise, suppose $w_{k_1} \sim w_{k_2}$ under $R_1$, then the corresponding elements $u_{i_1}, u_{i_2} \in \vu$ are equivalent under $R_1$, which is contradictory to the definition of $\vu$.

Next, we show that any given $w \in \vS_0$ is equivalent to some $w_k \le w$ under $R = R_1 \vee R_2$. Given $w$, there exists $u_i \le w$ and $v_j \le w$ which are equivalent to $w$ under $R_1$ and $R_2$ respectively. By definition of $R$, $w$ is equivalent to $u_i$ and $v_j$ under $R$ as well. If $u_i = v_j$, then $w \in \vw$ and the statement is true. Otherwise, suppose without loss of generality that $u_i < v_j$. Since $w \sim u_i$ under $R$, we just need to show that $u_i$ is equivalent to some $w_k \le u_i$. Repeat this process until some $u_i = v_j$, or we come to $u_1 = v_1 = \min \vS_0$, where the statement trivially holds.

The above two paragraphs show that $\vw$ is the minimum representative of $R$. [Still need to show the $R$ defined this way satisfies $R = R_1 \vee R_2$?] It then follows that for any $\vG \subseteq \vS_0$,
\[
\vF(\vG) = \vG \cap \vw = \vG \cap (\vu \cap \vv ) = (\vG \cap \vu) \cap \vv
= \vF_2(\vF_1(\vG)) .
\]

\end{proof}

%However, if each filter in implemented in a way such that it keeps a specific element as the representative, such as the lexical minimum of its equivalence class of each particular equivalence, then chaining them together may incorrectly drop out canonical guesses. This section discuss the conditions under which chaining them together will still produce correct results.


\chapter{Heuristic Strategies}

\section{Overview}

[motivation]

Heuristic strategies are interesting to study because they tend to uncover some interesting links between an optimal strategy and the immediate step. Good heuristics have an intuitive rationale as why the heuristic is constructed that way.

In this sense, the objective of a heuristic may not only be to yield an optimal solution quickly. [See e.g. neuwirth].

It is fair to expect that a tailored-heuristic to one configuration of the rules may not perform well in another configuration. This is the defect of tailored heuristics. (Show some examples)

See \cite{pepperdine10} for a list of heuristic functions.

\section{The Min-Max heuristic}

Knuth \cite{knuth76} published the first paper on Mastermind, where he introduced a heuristic strategy aiming at minimizing the worst-case number of remaining possibilities.

The same heuristic function first appeared in \cite{aleph71} for the Bulls and cows game, but was not elaborated.

Rationale: fewer remaining possibilities is better. We want to play safe and reduce the worst-case number of remaining possibilities. 

Note that this strategy does not need an assumption on the distribution of the secret. [see neu] (This, e.g. is suitable in mastermind with lie (or dynamic mastermind.))

Note: If two guesses yield the same worst-case partition, the second-to-worst partition size is compared, etc.

\section{The Min-Average heuristic}

Rationale: fewer remaining possibilities is better. We want to minimize the expected partition size.

\section{The Max-Entropy heuristic}

Rationale: we want a guess to provide as much information as possible as to determine what is the secret.

The entropy heuristic was first introduced by Neuwirth \cite{neuwirth81} for Mastermind and by Larmouth \cite{aleph71} for Bulls and cows. It is a theoretically advanced heuristic that scores a guess by the ``amount of information'' brought by its partitioning of the potential secret set. To be precise, this heuristic aims to maximize the \emph{entropy} of the partition, defined as
\[
H = -\sum_i \frac{n_i}{n} \log \frac{n_i}{n} .
\]
The base of the logarithm is not specified but it does not impact the choice. 

Rearranging terms, it can be written as
\begin{align}
H 
&= - \frac{1}{n} \left[ \sum_i n_i (\log n_i - \log n ) \right] \notag \\
&= - \frac{1}{n} \left( \sum_i n_i \log n_i \right) + \log n . \notag
\end{align}
When $n$ is fixed, maximizing $H$ is equivalent to minimizing the heuristic function
\[
h(P) = \sum_i n_i \log n_i .
\]
If we loosely interpret $\log n_i$ as an estimate of the number of further guesses needed for a partition of size $n_i$, then we can interpret the heuristic function (when divided by $n$) as an estimate of the expected number of further guesses needed. 

[Note: floating point precision?]

Heeffer \cite{heeffer07} tested various heuristic algorithms on the Mastermind game with 5 pegs and 8 colors, and found the entropy heuristic to perform the best.

See \href{http://en.wikipedia.org/wiki/Entropy\_(information\_theory)\#Further\_properties}{Wikipedia}.

Each feedback from a guess can be thought of as a "alphabet"
For p4c10n, there are 14 alphabets, but some letters are more likely to follow certain letters than others. However, such likelyhood depends on the guess chosen.

For example, if a guess partitions the possibility set into discrete partition, then all letters in that alphabet 

If alphabet is equally likely, then the entropy is maximized.


***************

Why entropy heuristic doesn't yield best (worst step) and (average step)?

The apparent underperformance of the entropy heuristic could be explained by noting the fact that there is a distinction between \emph{determining} the secret and \emph{revealing} the secret. Suppose we are left with 2 possibilities: 5678 and 7890. We can \emph{determine} the secret with one guess (e.g. 5678). However, to actually \emph{reveal} the secret, we need to make an extra guess (7890) if 5689 returns \fb{0}{2}. In total we need maximum 2 guesses and average 1.5 guesses.

However, from a information theory's perspective, the extra guess is totally redundant in that there is no uncertainty of the outcome: we know for sure that we will get \fb{4}{0} when we guess \cw{7890}. In fact, when we guess 5678, the entropy of the resulting partition \{(5678:4A0B),(7890:0A2B)\} is zero (ignoring the constant denominator), which means uncertainty removed.

The extra step to reveal the secret is necessary in the traditional human game because otherwise it's difficult to judge that the code breaker wins. On the other hand, the human game rules could be slightly modified to remove the need for the extra guess. Instead of required to \emph{reveal} the guess with a 4A0B feedback, the codebreaker is required to \emph{assert} the guess after a number of rounds. If the assertion is correct, he wins; if the assertion is wrong, he loses. The number of guesses one needs before making an assertion is equal to the number of steps one needs to determine the secret, and this number is consistent with an information-theory perspective.

To cope with subtle discrepancy of determining and revealing the secret, the entropy heuristic could be amended to distinguish the difference, though in this case the theory is not that sound. See [taiwan wang you] For example, Larmouth \cite{aleph71} used the following entropy heuristic \emph{with correction}:
\[
h'(P) = \sum_i n_i \log n_i - (2 \log 2) \delta(\fb{4}{0})  .
\]
The correction term applies when the partition contains \fb{4}{0}. However, how the coefficient $(2 \log 2)$ is derived is unclear.

Another issue with the entropy heuristic is that when computing the entropy, it only depends on the probablity of each partition (i.e. the size of each partition). This is because in entropy theory, it assumes that we know absolutely nothing about the underlying random variable (the secret), except which partition it resides in. Under this assumption, two partitions with the same size gives the same amount of information; for example, if partition A and partiton B both contain 3 possibilities, then if either case turns out to contain the secret, then we are equipped with the knowledge that we have 3 possibilities left, without any more knowledge.

However, in the scenario of Mastermind, the situation is different, because we have extra knowledge about the codewords apart from the size. Consider two partitions of the same size:

A = (1234, 1235, 1236), and

B = (1234, 1235, 2135)

Though both partitions contain the same number of elements, their information content is different; partition A has more uncertainty (in Mastermind sense) than partition B. This is because it requires at least 2 steps to determine the secret in A (this can be verified by an exhaustive search). However, partition B can be determined with one guess (for example, any one of the three secrets). Thus, with the extra knowledge not present in a vanilla entropy theory, partition A has more uncertainty.

Thus the assumptions of applying the entropy theory do not exactly hold. Consequenty, the strategy produced by entropy theory may not be as good as it apparently suggests. This may also mean that the logarithm base in computing the entropy may need to be adapted to the actual structure of the partition. However, if we continue this step recursively, then it is essentially an exhaustive strategy, in which case we don't need the heuristic any more.



\section{The Max-Parts Heuristic}

Rationale is problematic. Subject to choice of equal heuristic element. (We can perform a randomized test to permute the codeword list.)

The \maxpar{} heuristic was introduced by Kooi \cite{kooi05} who applied it to the Mastermind game and outperformed all other heuristic strategies. It proved to work well for a Mastermind game with 4 pegs and 6 colors, but didn't work well for one with 5 pegs and 8 colors.

The formula is
\[
h(P) = r ,
\]
where $r$ is the number of n(on-empty) partitions.

%\subsection{The Min-Steps Heuristic}

For Bulls and cows, the \maxpar{} strategy might have been tested in as early as 1969 by Ken Thompson \cite{ritchie01}. However, it is apparent that this strategy doesn't perform as well as the other heuristic strategies. See appendix [???] for an table.

\section{Other heuristics}

%http://mercury.webster.edu/aleshunas/Support\%20Materials/Analysis/Dowelll\%20-\%20Mastermind%20v2-0.doc

%Defeating Mastermind
%By Justin Dowell

%- WideDev
%- LongRect both hybrid strats

These are not good. Because we need a "rationale" for the heuristic. 

A few more heuristics have been tested by various authors. They are listed below for completeness. However, the rationale for each of the heuristics is not clear, so their performance may be expected to vary widely with the configuration of the rules.

[Move this to appendix]

For a large class of heuristics, the heuristic function is computed as the expectation of some function, $f$ of the partition size, optionally minus a correction term, $\lambda$ if the guess is in the possibilities. That is,
\[
h(P) = \frac{1}{n} \left[\sum_i n_i f(n_i) - \tau(\lambda) \right].
\]

The following list of functions appear in \cite{pepperdine10}:
\begin{center}
\begin{tabular}{c l l}
\hline
Name & $f$ & $\lambda$ \\
\hline
Modified entropy & $\log (1+n_i)$ & 0 \\
Landy's function & $L(n_i)$ where $L(n)$ is the solution of $x^x = n$ & 0 \\
Exponential asymptote & $1-e^{-n_i}$ & $(1-e^{-1})$ \\
square root & $\sqrt{n_i}$ & 1 \\
Logarithmic integral & $\text{li}(1+n_i)$ & $2 \cdot \text{li}(3)$ \\
\hline
\end{tabular}
\end{center}
More examples can be found in his document.

In addition, some hybrid strategies are used.

\section{Comparison of heuristics}

When several candidate guesses yield the same heuristic value, a choice must be made as to pick which one as the guess. Standard way is to choose the ``first'' candidate as it appears in the list, or the lexicographically minima. However, some evidence (where??) shows that the performance of a heuristic does depend on which choice is made. This is not ideal.




% -------------------------------- %






















% -------------------------------- %
\section{Optimal Strategies}

After examining a variety of heuristic strategies that perform variably under different configuration of rules, a natural question arise: for a given set of rules, is there an optimal strategy? The answer is ``yes''. In this chapter, we are going to explore the techniques that enable us to find such a strategy.

\subsection{Overview}

An \emph{optimal strategy} is a strategy that achieves a certain objective in an optimal manner. Three types of objectives are typical:

1) To minimize the worst-case number of guesses needed to reveal the secret.

2) To minimize the average number of guesses needed to reveal the secret.

3) To minimize the average number of guesses while keeping the worse-case steps to a minimum.

Note that in each case, we could replace ``reveal'' with ``determine'', which are subtly different from an information perspective (see 2.4). However the overall methodology will remain the same. Therefore in the following we will proceed with the above goals and give results to the ``determine'' version along the way.

For the first goal, the \minmax{} heuristic strategy (2.2) already provides an optimal solution, as it reveals all secrets with no more than 5 guesses and any strategy cannot use fewer [proof???]. So in this chapter we will focus on the second and third objectives.

Note that there may be other objectives for a strategy, such as minimizing the number of guesses evaluated. See, for example, Temporel and Kovacs (2003). However, those objectives are not studied in this chapter.

The theory to find an optimal strategy is simple. Since the number of possible secrets as well as sequence of (non-redundant) guesses are finite, the code breaker can employ a depth-first search to find out the optimal guessing strategy.

The optimal strategy minimizes the expected number of guesses, optionally subject to a maximum-guesses constraint. Finding such a strategy involves exhaustive search, and therefore is too slow to be suitable for real-time application. For more details on the implementation, see optimal strategies.

[show the search scale of the problem]

[show that an optimal strategy is not unique]

\subsection{Obviously-optimal guesses}

Some definitions. Partitions, feedback count, etc.



While finding an optimal strategy for the general game is complex, in certain cases it's easy. For example, when there's only one possibility left, we should guess it. When there are only two possibilities left, we should guess (either) one of them. [these appear in Neuwirth 82]. When there are more than two possibilities, it's still possible.

An obviously-optimal guess is an optimal guess that doesn't require too much effort to identify. Depending on the techniques used to identify such, the "obvious"-ty could vary. Here we use the technique introduced by \cite{koyama93}. 

[Definition.] An obviously-optimal guess is a guess that partitions the remaining possibilities into discrete cells, i.e.\ where every cell contains exactly one element. 

If such a guess exists and comes from the possibility set, then it is optimal because it reveals one potential secret (itself) in the immediate step and reveals all the other potential secrets in two steps. It is easy to see that no other strategy could do better. If no such guess exists in the possibility set but one exists outside the possibility set, then that one is optimal because it reveals all secrets in two steps. 

Note that an obviously-optimal guess is fairly generic about the goal -- it is optimal both in terms of the worst-case number of steps and the expected number of steps to determine or reveal the secret.

A necessary condition for an obviously-optimal guess to exist is that the number of remaining possibilities does not exceed the number of distinct feedbacks. For a game with $p$ pegs, the number of distinct feedbacks is $p(p+3)/2$. For example, in a four-peg game, there can be at most 14 secrets left for an obviously-optimal guess to exist. This is a useful check in practice to reduce unnecessary efforts to find an obviously optimal guess.

Note also that in practice we may only want to check in the remaining possibilities (so that the effort is minimized). In turns out that if we check outside the remaining possibilities, it is equivalent to a full-run of a heuristic function, as we show below.

It turns out (not so surprisingly) that the heuristics introduced in the previous chapter will yield an obviously-optimal guess when one exists. We only need to show that the partition of an obviously-optimal guess (which we will call an \emph{obviously optimal partition} and denote by $Q$ below) achieves the lowest possible heuristic value.

Let $P$ denote any given partition. Let $k$ denote the number of (non-empty) cells in $P$. Let $n_i$ denote the number of elements in the $i$-th cell. Let $n = \sum_{i=1}^k$ denote the total number of elements, which is invariant across different $P$. Finally, let $Q$ denote the partition of an obviously-optimal guess, i.e. one with all singleton cells.

\paragraph{Min-Max} 
The \minmax{} heuristic value of an obviously optimal partition is one. This is the minimum value of the heuristic function.
\[
h(P) = \max_{1 \le i \le n} n_i \ge 1 = h(Q).
\]

\paragraph{Min-Avg}
The \minavg{} heuristic value of an obviously optimal partition is one. This is the minimum value of the function.
\[
h(P) = \sum_{i=1}^k \frac{n_i}{n} n_i \ge \min_{1 \le i \le k} n_i \ge 1 = h(Q).
\]

\paragraph{Max-Entropy}
The \maxent{} heuristic value of an obviously optimal partition is ?. This is the maximum value of the function.
\[
h(P) = - \sum_{i=1}^k \frac{n_i}{n} \log \frac{n_i}{n} = ?
\]

\paragraph{Max-Parts}
The \maxpar{} heuristic value of an obviously optimal partition is $n$. This is the maximum value of the function.
\[
h(P) = k \le n = h(Q).
\]

Since all four heuristics introduced yield an obviously optimal guess when one exists, we can insert the step to find an optimal guess into the strategy as a shortcut to save computation time, knowing that this will not alter the output of the heuristic strategy.

\subsection{Equivalence of guesses}

Isomorphism of guesses

\subsubsection{Color equivalence}

\subsubsection{Constraint equivalence}

Constraint/filter equivalence

\subsubsection{Partition equivalence}

State/Partition Equivalence

\subsection{Search space pruning}

\subsection{Other techniques}

(e.g. two-phase optimization, hash collision group)

\subsection{Using a pre-built strategy tree}

\subsection{Extended/Adaptive strategy tree}

i.e. the tree not only contains guesses along the chosen strategy path, but also includes guesses if the user made a non-optimal guess halfway. The tree size in this case is much larger, and we must use isomorphism to detect the symmetry.

 

\chapter{Randomized Strategies}

\section{Mastermind Satisfiability Problem}

Mastermind Satisfiability Problem is NP-Complete.

[know nothing about this]
\section{Variations}

Other Topics, Related Topics

\subsection{Static Mastermind}

\subsection{Dynamic Mastermind}

In the standard Mastermind game, all that the code-maker does is to set up a secret at the beginning of the game, and then \emph{passively} responds to the guesses made by the code-maker. The role played by the code-maker is rather boring.

To make the code-maker's role more interesting (and challenging), in a \emph{dynamic} Mastermind game, introduced by Bestavros and Belal \cite{bestavros86}, the codemaker is allowed to silently change the secret during the course of the game, as long as the secret is consistent with all the guesses and feedbacks so far. 

For example, suppose the code-maker initially holds the secret 1234. Suppose that in the first round, the code-breaker makes the guess 1234 outright. Under the standard rules, the code-maker is obliged to respond with 4A0B and surrender the game. However, under the \emph{dynamic} rules, the code-maker is allowed to silently change the secret to, say, 3456, and reply with the feedback 0A2B. The game continues until the code-maker finally reveals the secret.

It is easy to see that under either the standard rules or the dynamic rules, the code-maker will eventually lose because there are a finite number of codewords. However, under the dynamic rules, the code-maker is able to prolong the game for more steps. 

In fact, it can be shown that when both sides play optimally, the game will require exactly [7??] rounds to finish. That is, after six rounds of guesses and responses, there is only one codeword left that conforms to all the constraints so far; the code-breaker will then guess this codeword and the code-maker has to respond with 4A0B.

[Show this using exhaustive search]

[Compare different combinations of heuristic strategies used by the code-maker and code-breaker]


\subsection{Mastermind with a lie}


\chapter{C++ Implementation}

\section{Overview}

The program in this project plays the role of the code breaker. The goal is to find out the secret using as few guesses as possible. The program supports both the Mastermind rules and the GuessNumber rules, subject to the following limits:

Maximum number of colors (defined by MM\_MAX\_COLORS): 10.

Maximum number of pegs (defined by MM\_MAX\_PEGS): 6.

Program Optimization Techniques

While the code-breaking algorithms are quite straightforward, much effort of this project has been put to optimize the performance of a real-time code breaker. Some of the hot-spots are identified by the profiler. The major points of optimization are described below. Most of the optimized routines are implemented in standalone source files for clarity.

Search space pruning. While all codewords are candidates for making a guess, some of them are equivalent in terms of bringing new information. For example, in the first round of a Number Guessing game, any guess works the same. Aware of this, we implement pruning by classifying digits into three classes: impossible, unguessed, and the rest. After each round of feedback, we update the list of distinct guesses (in terms of bringing information) and search within this list only. This reduces the search space significantly.

Codeword comparison. This is the most intensive operation in the program, accounting for 40\% of all CPU time. The program uses SSE2 instructions (implemented via compiler intrinsics) to compare each pair of codewords in four instructions.

Feedback frequency counting. The heuristic code breaker relies heavily on these routines to count statistics on partitions. This is an intensive operation which accounts for about 20\% of all CPU time. The program uses an ASM implementation to maximize performance. See Frequency.cpp.

\section{Data Structure}

\subsection{Codeword}

\subsection{Feedback}

\subsection{Strategy Tree}

\section{Basic operations}

\subsection{Codeword comparison}

\subsection{Frequency table generation}

\subsection{Codeword set partitioning}

\section{Implementing Heuristic Strategies}

 Obvious guesses
 
\section{Implementing Optimal Strategies}
 
\section{Implementing randomized strategies}


\appendix
\section{Appendix}

\subsection{Mathematical Background}

\subsubsection{Information entropy}

entropy

\subsubsection{Equivalence relation}

See \url{http://en.wikipedia.org/wiki/Equivalence\_relation}.

\subsubsection{Permutation}

[permutation notations, properties]

[equivalence relation, equivalence class, etc.]

A \emph{permutation} is a bijection (one-to-one mapping) of a set onto itself.\footnote{
More details can be found at \url{http://en.wikipedia.org/wiki/Permutation}.}
For example, consider a set containing six elements. For convenience, label each element with an index starting from one. A possible permutation is the following:
\[
\begin{pmatrix}
1 & 2 & 3 & 4 & 5 & 6 \\
5 & 2 & 1 & 6 & 3 & 4
\end{pmatrix} ,
\]
where the first row displays the elements in the set, and the second row displays the \emph{images} of the elements under the permutation, i.e.\ the element that each element is mapped to mapped to.

The above notation for the permutation can be abbreviated into one line by only keeping the second row, which becomes $(5 2 1 6 3 4)$.

A permutation can be decomposed into a \emph{product} of disjoint \emph{cycles}, which partitions the elements of the set into parts where the elements in each part can be permuted independently to complete the overall mapping. For example, the above permutation can be written as
\[
\begin{pmatrix}
1 & 2 & 3 & 4 & 5 & 6 \\
5 & 2 & 1 & 6 & 3 & 4
\end{pmatrix} 
= (1 5 3) (2) (4 6) .
\]
It is easy to see that the cycles can be commuted and the elements in a cycle can be rotated without changing the overall permutation. Up to these differences, such decomposition is unique.

A permutation can be inverted. To find the inverse of a permutation, simply exchange the two roles in the notation and then sort the upper row. In the above example, the inverse permutation is
\[
\begin{pmatrix}
1 & 2 & 3 & 4 & 5 & 6 \\
3 & 2 & 5 & 6 & 1 & 4
\end{pmatrix} .
\]

%Two permutations of the same size can be compounded to form a composite permutation. Let @c P be the composite of permutation
% * <code>P<sub>1</sub></code> and <code>P<sub>2</sub></code>.
% * The effect of applying @c P is equivalent to first applying
% * <code>P<sub>1</sub></code> followed by applying <code>P<sub>2</sub></code>.
% * The notation to write a composite permutation can be confusing,
% * so we omit it here.

A \emph{partial permutation} on a set is a bijection between two subsets of it.\footnote{For more details, see \url{http://www.maths.qmul.ac.uk/~pjc/odds/partial.pdf}.}
For example, a partial permutation of the above (complete) permutation could be
\[
\begin{pmatrix}
1 & 2 & 3 & 4 & 5 & 6 \\
* & 2 & * & * & 3 & 4
\end{pmatrix} ,
\]
where the asterisks denote unmapped elements, sometimes known as ``holes'' of the permutation. 

It is easy to see that any partial permutation can be \emph{extended} to form a complete permutation. However such extension is not unique. For example, there are $3! = 6$ ways to extend the above partial permutation, one of which that differs from the original example could be
\[
\begin{pmatrix}
1 & 2 & 3 & 4 & 5 & 6 \\
1 & 2 & 5 & 6 & 3 & 4
\end{pmatrix} .
\]

\subsubsection{Graph isomorphism}

\subsection{Optimal strategy tree for Mastermind}

With repetition, without repetition

\subsection{Comparison of different strategies}

display a set diff of the strategy trees

\nocite{*}
\bibliographystyle{plain}
\bibliography{mastermind}

\end{document}
