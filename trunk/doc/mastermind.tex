%\documentclass[8pt,a5paper]{article}
\documentclass{report}
\usepackage{amsmath}
\usepackage{amsthm}
\usepackage{hyperref}
\hypersetup{
	pdftitle={Mastermind Strategies},
	pdfkeywords={Mastermind, heuristic},
	colorlinks=true,
	pdfstartview={FitH},
	linkcolor=black,
	citecolor=blue,
	urlcolor=black,
}
%\usepackage[small,textfont=it]{caption}
\usepackage[small]{caption}

\setlength{\parskip}{1ex plus 0.5ex minus 0.2ex}

\newcommand{\given}{\,|\,}
\newcommand{\minmax}{\texttt{min-max}}
\newcommand{\minavg}{\texttt{min-avg}}
\newcommand{\maxent}{\texttt{max-entropy}}
\newcommand{\maxpar}{\texttt{max-parts}}
\newcommand{\cw}[1]{\texttt{#1}}
\newcommand{\fb}[2]{\texttt{{#1}A{#2}B}}
%\newcommand{\u}[1]{\underline{#1}}

\theoremstyle{definition}
\newtheorem*{definition}{Definition}
%\newtheorem{definition}{Definition}
\newtheorem*{note}{Note}

\begin{document}

\title{Introduction to Mastermind Strategies}
% Alternative book name:
%  - (Practical) Mastermind Strategies
%  - Mastering Mastermind
\author{Author}
\date{2012}
\maketitle

\begin{abstract}

This document provides a systematic review of the main strategies for playing the Mastermind game and some of its variants, including the well-studied Bulls and Cows game. Extensive literature review is conducted and a sample program is provided at the end.

Chapter 1 gives an overview of the Mastermind game and its strategies. Basic concepts and properties about the game are formalized such as a strategy tree. At the end of the chapter, the (standard) notations used to represent a strategy tree in literature is introduced.

Chapter 2 describes several heuristic strategies, including \minmax{}, \minavg{}, \maxent{} and \maxpar{}. The rationale of each heuristic as well as their performance and limitations are analyzed. At the end of the chapter lists a few minor heuristics together with a comparison of their performance under common settings of game.

Chapter 3 discusses the techniques that are commonly employed to search for an optimal strategy. Of main importance are guess equivalence detection and Alpha-Beta pruning. We introduce three types of equivalence criteria varying in degree of complexity and performance. [We (will) also discuss pruning.]

Chapter 4 provides an overview of randomized strategies aimed at solving large-scale games efficiently. The need for a randomized strategy is justified by the fact that the Mastermind Satisfiability Problem is NP-complete. [To be completed]

Chapter 5 introduces a few variants of the game and discusses the corresponding strategies. Static Mastermind is a game where all the guesses are made all at once. Dynamic Mastermind is where the code-maker can silently change the secret during the course of the game. Finally, Mastermind with a lie is where at most one of the response can be fake.

Chapter 6 details a computer program implementation of the strategies discussed above.

\end{abstract}


\tableofcontents

%[*] entries marked with an asterisk are advanced topics. Skipping these topics doesn't affect a basic understanding of the subject. However they are needed for a practical implementation of the strategies.

% Chapter 1. Introduction
%   - Overview (history, rules, variations)
%   - (Interactive) Examples
%   - Overview of strategies (introduce the strategies, and introduce a table that describes how many codewords need to be solved in how many steps)
% [secrets be revealed]
%   - A Simple Strategy
%   - Obviously-Optimal guesses
%   - Strategy tree

% Chapter 2. Heuristic Strategies
%   - Overview [How it works]
%   - The MinMax heuristic
%   - The Minavg heuristic
%   - The Max Entropy heuristic
%   - The Max Parts heuristic
%   - The Min Steps heuristic
%   - Hybrid heuristics (e.g. WideDev, LongRect)
%   - Comparison of heuristics

% Chapter 3. Optimal Strategies
%   - Overview
%   - Equivalence of guesses [?] (Filter Equivalence)
%     - Color Equivalence
%     - Constraint Equivalence
%     - Partition Equivalence
%   - Isomorphism of guesses [*] (State/Partition Equivalence)
%   - Search space pruning [*]
%   - Using a pre-built strategy tree [*]
%   - Extended/Adaptive strategy tree [*] (i.e. the tree not only contains guesses along the chosen strategy path, but also includes guesses if the user made a non-optimal guess halfway. The tree size in this case is much larger, and we must use isomorphism to detect the symmetry.)

% Chapter 4. Randomized Strategies
%   - Mastermind Satisfying Problem is NP-Complete
% [know nothing about this]

% Chapter 5. Variations (Other Topics, Related Topics)
%   - With Lies
%   - Static Mastermind
%   - Dynamic Mastermind
%   - Other improvements (e.g. two-phase optimization, hash collision group)

% Chapter 6. C++ Implementation
%   - Overview
%   - Data Structure
%     - Codeword
%     - Feedback
%     - Strategy Tree
%   - Basic operations
%     - Codeword comparison
%     - Frequency table generation
%     - Codeword set partitioning
%   - Implementing Heuristic Strategies
%     - Obvious guesses
%   - Implementing Optimal Strategies
%   - Implementing randomized strategies

% Reference

% Appendix
%   - Entropy theory
%   - Graph theory (in particular, graph isophosim)
%   - Optimial strategy listing for Mastermind 
%   - Optimal strategy listing for Guess-Number
%   - Comparison of different strategiesStrategy tree

\section{Introduction}

\subsection{Overview}

[history, rules, variations, etc]

Mastermind is a game played by two players, the \emph{code maker} and the \emph{code breaker}. At the beginning of the game, the code maker thinks of a secret codeword with $p$ pegs and $c$ colors. Then the code breaker begins making guesses.

For each guess, the code maker responds with a \emph{feedback} in the form of xAyB to describe how close the guess is to the secret. Here $x$ is the number of correct colors in the correct pegs, and $y$ is the number of correct colors in the wrong pegs. The game continues until the code breaker finds the secret.

The standard Mastermind game is played on a board with four pegs and using six colors. A codeword can contain the same color more than once. A popular variation of the rules is called GuessNumber, also known as Bulls-n-Cows. Under these rules, no repetition of colors is allowed in the codeword.

\subsection{Examples}

(Interactive) Examples

\subsection{Overview of strategies}

[introduce the strategies, and introduce a table that describes how many codewords need to be solved in how many steps). [secrets be revealed]

There are three types of strategies for the code breaker:

Simple strategy. (Ch02) The code breaker just makes a random guess, as long as the guess will bring some information. A natural choice could be the first codeword from the remaining possibilities. These strategies obviously perform poorly in that it usually takes many steps to find the secret, but they are simple and fast and thus can be used as a benchmark for other more sophisticated strategies.

Heuristic strategy. (Ch03) The code breaker evaluates each potential guess according to some scoring criteria, and picks the one that gets the highest score. These strategies are fast enough for real-time games, and performs quite well. Most research in this field focuses on finding a good heuristic criteria, and a number of intuitive and well-performing heuristics have been proposed. For more details, see heuristic strategies.

Optimal strategy. (Ch04) Since the number of possible secrets as well as sequence of guesses are finite, the code breaker can employ a depth-first search to find out the optimal guessing strategy. The optimal strategy minimizes the expected number of guesses, optionally subject to a maximum-guesses constraint. Finding such a strategy involves exhaustive search, and therefore is too slow to be suitable for real-time application. For more details on the implementation, see optimal strategies.

\subsection{A Simple Strategy}

Chooses the \emph{first} possibility. Here first depends on the order of the list. We can let it be lexicographical first. 

However, an algorithm that depends on lexicographical ordering is not very ideal. See comparision of heuristics in chapter 2 for a discussion.
 
\subsection{Strategy tree}

A standard strategy tree contains the strategy starting from the initial state.

An extended (or adaptive, full) strategy tree contains a strategy for every possible state of the game.

Introduce the notations (Irving) to aid the reader in reading the classical papers.
  
\subsection{Lower-bound of the number of guesses}

% -------------------------------- %
























% -------------------------------- %
\chapter{Heuristic Strategies}

\section{Overview}

[motivation]

Heuristic strategies are interesting to study because they tend to uncover some interesting links between an optimal strategy and the immediate step. Good heuristics have an intuitive rationale as why the heuristic is constructed that way.

In this sense, the objective of a heuristic may not only be to yield an optimal solution quickly. [See e.g. neuwirth].

It is fair to expect that a tailored-heuristic to one configuration of the rules may not perform well in another configuration. This is the defect of tailored heuristics. (Show some examples)

See \cite{pepperdine10} for a list of heuristic functions.

\section{The Min-Max heuristic}

Knuth \cite{knuth76} published the first paper on Mastermind, where he introduced a heuristic strategy aiming at minimizing the worst-case number of remaining possibilities.

The same heuristic function first appeared in \cite{aleph71} for the Bulls and cows game, but was not elaborated.

Rationale: fewer remaining possibilities is better. We want to play safe and reduce the worst-case number of remaining possibilities. 

Note that this strategy does not need an assumption on the distribution of the secret. [see neu] (This, e.g. is suitable in mastermind with lie (or dynamic mastermind.))

Note: If two guesses yield the same worst-case partition, the second-to-worst partition size is compared, etc.

\section{The Min-Average heuristic}

Rationale: fewer remaining possibilities is better. We want to minimize the expected partition size.

\section{The Max-Entropy heuristic}

Rationale: we want a guess to provide as much information as possible as to determine what is the secret.

The entropy heuristic was first introduced by Neuwirth \cite{neuwirth81} for Mastermind and by Larmouth \cite{aleph71} for Bulls and cows. It is a theoretically advanced heuristic that scores a guess by the ``amount of information'' brought by its partitioning of the potential secret set. To be precise, this heuristic aims to maximize the \emph{entropy} of the partition, defined as
\[
H = -\sum_i \frac{n_i}{n} \log \frac{n_i}{n} .
\]
The base of the logarithm is not specified but it does not impact the choice. 

Rearranging terms, it can be written as
\begin{align}
H 
&= - \frac{1}{n} \left[ \sum_i n_i (\log n_i - \log n ) \right] \notag \\
&= - \frac{1}{n} \left( \sum_i n_i \log n_i \right) + \log n . \notag
\end{align}
When $n$ is fixed, maximizing $H$ is equivalent to minimizing the heuristic function
\[
h(P) = \sum_i n_i \log n_i .
\]
If we loosely interpret $\log n_i$ as an estimate of the number of further guesses needed for a partition of size $n_i$, then we can interpret the heuristic function (when divided by $n$) as an estimate of the expected number of further guesses needed. 

[Note: floating point precision?]

Heeffer \cite{heeffer07} tested various heuristic algorithms on the Mastermind game with 5 pegs and 8 colors, and found the entropy heuristic to perform the best.

See \href{http://en.wikipedia.org/wiki/Entropy\_(information\_theory)\#Further\_properties}{Wikipedia}.

Each feedback from a guess can be thought of as a "alphabet"
For p4c10n, there are 14 alphabets, but some letters are more likely to follow certain letters than others. However, such likelyhood depends on the guess chosen.

For example, if a guess partitions the possibility set into discrete partition, then all letters in that alphabet 

If alphabet is equally likely, then the entropy is maximized.


***************

Why entropy heuristic doesn't yield best (worst step) and (average step)?

The apparent underperformance of the entropy heuristic could be explained by noting the fact that there is a distinction between \emph{determining} the secret and \emph{revealing} the secret. Suppose we are left with 2 possibilities: 5678 and 7890. We can \emph{determine} the secret with one guess (e.g. 5678). However, to actually \emph{reveal} the secret, we need to make an extra guess (7890) if 5689 returns \fb{0}{2}. In total we need maximum 2 guesses and average 1.5 guesses.

However, from a information theory's perspective, the extra guess is totally redundant in that there is no uncertainty of the outcome: we know for sure that we will get \fb{4}{0} when we guess \cw{7890}. In fact, when we guess 5678, the entropy of the resulting partition \{(5678:4A0B),(7890:0A2B)\} is zero (ignoring the constant denominator), which means uncertainty removed.

The extra step to reveal the secret is necessary in the traditional human game because otherwise it's difficult to judge that the code breaker wins. On the other hand, the human game rules could be slightly modified to remove the need for the extra guess. Instead of required to \emph{reveal} the guess with a 4A0B feedback, the codebreaker is required to \emph{assert} the guess after a number of rounds. If the assertion is correct, he wins; if the assertion is wrong, he loses. The number of guesses one needs before making an assertion is equal to the number of steps one needs to determine the secret, and this number is consistent with an information-theory perspective.

To cope with subtle discrepancy of determining and revealing the secret, the entropy heuristic could be amended to distinguish the difference, though in this case the theory is not that sound. See [taiwan wang you] For example, Larmouth \cite{aleph71} used the following entropy heuristic \emph{with correction}:
\[
h'(P) = \sum_i n_i \log n_i - (2 \log 2) \delta(\fb{4}{0})  .
\]
The correction term applies when the partition contains \fb{4}{0}. However, how the coefficient $(2 \log 2)$ is derived is unclear.

Another issue with the entropy heuristic is that when computing the entropy, it only depends on the probablity of each partition (i.e. the size of each partition). This is because in entropy theory, it assumes that we know absolutely nothing about the underlying random variable (the secret), except which partition it resides in. Under this assumption, two partitions with the same size gives the same amount of information; for example, if partition A and partiton B both contain 3 possibilities, then if either case turns out to contain the secret, then we are equipped with the knowledge that we have 3 possibilities left, without any more knowledge.

However, in the scenario of Mastermind, the situation is different, because we have extra knowledge about the codewords apart from the size. Consider two partitions of the same size:

A = (1234, 1235, 1236), and

B = (1234, 1235, 2135)

Though both partitions contain the same number of elements, their information content is different; partition A has more uncertainty (in Mastermind sense) than partition B. This is because it requires at least 2 steps to determine the secret in A (this can be verified by an exhaustive search). However, partition B can be determined with one guess (for example, any one of the three secrets). Thus, with the extra knowledge not present in a vanilla entropy theory, partition A has more uncertainty.

Thus the assumptions of applying the entropy theory do not exactly hold. Consequenty, the strategy produced by entropy theory may not be as good as it apparently suggests. This may also mean that the logarithm base in computing the entropy may need to be adapted to the actual structure of the partition. However, if we continue this step recursively, then it is essentially an exhaustive strategy, in which case we don't need the heuristic any more.



\section{The Max-Parts Heuristic}

Rationale is problematic. Subject to choice of equal heuristic element. (We can perform a randomized test to permute the codeword list.)

The \maxpar{} heuristic was introduced by Kooi \cite{kooi05} who applied it to the Mastermind game and outperformed all other heuristic strategies. It proved to work well for a Mastermind game with 4 pegs and 6 colors, but didn't work well for one with 5 pegs and 8 colors.

The formula is
\[
h(P) = r ,
\]
where $r$ is the number of n(on-empty) partitions.

%\subsection{The Min-Steps Heuristic}

For Bulls and cows, the \maxpar{} strategy might have been tested in as early as 1969 by Ken Thompson \cite{ritchie01}. However, it is apparent that this strategy doesn't perform as well as the other heuristic strategies. See appendix [???] for an table.

\section{Other heuristics}

%http://mercury.webster.edu/aleshunas/Support\%20Materials/Analysis/Dowelll\%20-\%20Mastermind%20v2-0.doc

%Defeating Mastermind
%By Justin Dowell

%- WideDev
%- LongRect both hybrid strats

These are not good. Because we need a "rationale" for the heuristic. 

A few more heuristics have been tested by various authors. They are listed below for completeness. However, the rationale for each of the heuristics is not clear, so their performance may be expected to vary widely with the configuration of the rules.

[Move this to appendix]

For a large class of heuristics, the heuristic function is computed as the expectation of some function, $f$ of the partition size, optionally minus a correction term, $\lambda$ if the guess is in the possibilities. That is,
\[
h(P) = \frac{1}{n} \left[\sum_i n_i f(n_i) - \tau(\lambda) \right].
\]

The following list of functions appear in \cite{pepperdine10}:
\begin{center}
\begin{tabular}{c l l}
\hline
Name & $f$ & $\lambda$ \\
\hline
Modified entropy & $\log (1+n_i)$ & 0 \\
Landy's function & $L(n_i)$ where $L(n)$ is the solution of $x^x = n$ & 0 \\
Exponential asymptote & $1-e^{-n_i}$ & $(1-e^{-1})$ \\
square root & $\sqrt{n_i}$ & 1 \\
Logarithmic integral & $\text{li}(1+n_i)$ & $2 \cdot \text{li}(3)$ \\
\hline
\end{tabular}
\end{center}
More examples can be found in his document.

In addition, some hybrid strategies are used.

\section{Comparison of heuristics}

When several candidate guesses yield the same heuristic value, a choice must be made as to pick which one as the guess. Standard way is to choose the ``first'' candidate as it appears in the list, or the lexicographically minima. However, some evidence (where??) shows that the performance of a heuristic does depend on which choice is made. This is not ideal.




% -------------------------------- %






















% -------------------------------- %
\section{Optimal Strategies}

After examining a variety of heuristic strategies that perform variably under different configuration of rules, a natural question arise: for a given set of rules, is there an optimal strategy? The answer is ``yes''. In this chapter, we are going to explore the techniques that enable us to find such a strategy.

\subsection{Overview}

An \emph{optimal strategy} is a strategy that achieves a certain objective in an optimal manner. Three types of objectives are typical:

1) To minimize the worst-case number of guesses needed to reveal the secret.

2) To minimize the average number of guesses needed to reveal the secret.

3) To minimize the average number of guesses while keeping the worse-case steps to a minimum.

Note that in each case, we could replace ``reveal'' with ``determine'', which are subtly different from an information perspective (see 2.4). However the overall methodology will remain the same. Therefore in the following we will proceed with the above goals and give results to the ``determine'' version along the way.

For the first goal, the \minmax{} heuristic strategy (2.2) already provides an optimal solution, as it reveals all secrets with no more than 5 guesses and any strategy cannot use fewer [proof???]. So in this chapter we will focus on the second and third objectives.

Note that there may be other objectives for a strategy, such as minimizing the number of guesses evaluated. See, for example, Temporel and Kovacs (2003). However, those objectives are not studied in this chapter.

The theory to find an optimal strategy is simple. Since the number of possible secrets as well as sequence of (non-redundant) guesses are finite, the code breaker can employ a depth-first search to find out the optimal guessing strategy.

The optimal strategy minimizes the expected number of guesses, optionally subject to a maximum-guesses constraint. Finding such a strategy involves exhaustive search, and therefore is too slow to be suitable for real-time application. For more details on the implementation, see optimal strategies.

[show the search scale of the problem]

[show that an optimal strategy is not unique]

\subsection{Obviously-optimal guesses}

Some definitions. Partitions, feedback count, etc.



While finding an optimal strategy for the general game is complex, in certain cases it's easy. For example, when there's only one possibility left, we should guess it. When there are only two possibilities left, we should guess (either) one of them. [these appear in Neuwirth 82]. When there are more than two possibilities, it's still possible.

An obviously-optimal guess is an optimal guess that doesn't require too much effort to identify. Depending on the techniques used to identify such, the "obvious"-ty could vary. Here we use the technique introduced by \cite{koyama93}. 

[Definition.] An obviously-optimal guess is a guess that partitions the remaining possibilities into discrete cells, i.e.\ where every cell contains exactly one element. 

If such a guess exists and comes from the possibility set, then it is optimal because it reveals one potential secret (itself) in the immediate step and reveals all the other potential secrets in two steps. It is easy to see that no other strategy could do better. If no such guess exists in the possibility set but one exists outside the possibility set, then that one is optimal because it reveals all secrets in two steps. 

Note that an obviously-optimal guess is fairly generic about the goal -- it is optimal both in terms of the worst-case number of steps and the expected number of steps to determine or reveal the secret.

A necessary condition for an obviously-optimal guess to exist is that the number of remaining possibilities does not exceed the number of distinct feedbacks. For a game with $p$ pegs, the number of distinct feedbacks is $p(p+3)/2$. For example, in a four-peg game, there can be at most 14 secrets left for an obviously-optimal guess to exist. This is a useful check in practice to reduce unnecessary efforts to find an obviously optimal guess.

Note also that in practice we may only want to check in the remaining possibilities (so that the effort is minimized). In turns out that if we check outside the remaining possibilities, it is equivalent to a full-run of a heuristic function, as we show below.

It turns out (not so surprisingly) that the heuristics introduced in the previous chapter will yield an obviously-optimal guess when one exists. We only need to show that the partition of an obviously-optimal guess (which we will call an \emph{obviously optimal partition} and denote by $Q$ below) achieves the lowest possible heuristic value.

Let $P$ denote any given partition. Let $k$ denote the number of (non-empty) cells in $P$. Let $n_i$ denote the number of elements in the $i$-th cell. Let $n = \sum_{i=1}^k$ denote the total number of elements, which is invariant across different $P$. Finally, let $Q$ denote the partition of an obviously-optimal guess, i.e. one with all singleton cells.

\paragraph{Min-Max} 
The \minmax{} heuristic value of an obviously optimal partition is one. This is the minimum value of the heuristic function.
\[
h(P) = \max_{1 \le i \le n} n_i \ge 1 = h(Q).
\]

\paragraph{Min-Avg}
The \minavg{} heuristic value of an obviously optimal partition is one. This is the minimum value of the function.
\[
h(P) = \sum_{i=1}^k \frac{n_i}{n} n_i \ge \min_{1 \le i \le k} n_i \ge 1 = h(Q).
\]

\paragraph{Max-Entropy}
The \maxent{} heuristic value of an obviously optimal partition is ?. This is the maximum value of the function.
\[
h(P) = - \sum_{i=1}^k \frac{n_i}{n} \log \frac{n_i}{n} = ?
\]

\paragraph{Max-Parts}
The \maxpar{} heuristic value of an obviously optimal partition is $n$. This is the maximum value of the function.
\[
h(P) = k \le n = h(Q).
\]

Since all four heuristics introduced yield an obviously optimal guess when one exists, we can insert the step to find an optimal guess into the strategy as a shortcut to save computation time, knowing that this will not alter the output of the heuristic strategy.

\subsection{Equivalence of guesses}

Isomorphism of guesses

\subsubsection{Color equivalence}

\subsubsection{Constraint equivalence}

Constraint/filter equivalence

\subsubsection{Partition equivalence}

State/Partition Equivalence

\subsection{Search space pruning}

\subsection{Other techniques}

(e.g. two-phase optimization, hash collision group)

\subsection{Using a pre-built strategy tree}

\subsection{Extended/Adaptive strategy tree}

i.e. the tree not only contains guesses along the chosen strategy path, but also includes guesses if the user made a non-optimal guess halfway. The tree size in this case is much larger, and we must use isomorphism to detect the symmetry.

 

\chapter{Randomized Strategies}

\section{Mastermind Satisfiability Problem}

Mastermind Satisfiability Problem is NP-Complete.

[know nothing about this]
\section{Variations}

Other Topics, Related Topics

\subsection{Static Mastermind}

\subsection{Dynamic Mastermind}

In the standard Mastermind game, all that the code-maker does is to set up a secret at the beginning of the game, and then \emph{passively} responds to the guesses made by the code-maker. The role played by the code-maker is rather boring.

To make the code-maker's role more interesting (and challenging), in a \emph{dynamic} Mastermind game, introduced by Bestavros and Belal \cite{bestavros86}, the codemaker is allowed to silently change the secret during the course of the game, as long as the secret is consistent with all the guesses and feedbacks so far. 

For example, suppose the code-maker initially holds the secret 1234. Suppose that in the first round, the code-breaker makes the guess 1234 outright. Under the standard rules, the code-maker is obliged to respond with 4A0B and surrender the game. However, under the \emph{dynamic} rules, the code-maker is allowed to silently change the secret to, say, 3456, and reply with the feedback 0A2B. The game continues until the code-maker finally reveals the secret.

It is easy to see that under either the standard rules or the dynamic rules, the code-maker will eventually lose because there are a finite number of codewords. However, under the dynamic rules, the code-maker is able to prolong the game for more steps. 

In fact, it can be shown that when both sides play optimally, the game will require exactly [7??] rounds to finish. That is, after six rounds of guesses and responses, there is only one codeword left that conforms to all the constraints so far; the code-breaker will then guess this codeword and the code-maker has to respond with 4A0B.

[Show this using exhaustive search]

[Compare different combinations of heuristic strategies used by the code-maker and code-breaker]


\subsection{Mastermind with a lie}


\chapter{C++ Implementation}

\section{Overview}

The program in this project plays the role of the code breaker. The goal is to find out the secret using as few guesses as possible. The program supports both the Mastermind rules and the GuessNumber rules, subject to the following limits:

Maximum number of colors (defined by MM\_MAX\_COLORS): 10.

Maximum number of pegs (defined by MM\_MAX\_PEGS): 6.

Program Optimization Techniques

While the code-breaking algorithms are quite straightforward, much effort of this project has been put to optimize the performance of a real-time code breaker. Some of the hot-spots are identified by the profiler. The major points of optimization are described below. Most of the optimized routines are implemented in standalone source files for clarity.

Search space pruning. While all codewords are candidates for making a guess, some of them are equivalent in terms of bringing new information. For example, in the first round of a Number Guessing game, any guess works the same. Aware of this, we implement pruning by classifying digits into three classes: impossible, unguessed, and the rest. After each round of feedback, we update the list of distinct guesses (in terms of bringing information) and search within this list only. This reduces the search space significantly.

Codeword comparison. This is the most intensive operation in the program, accounting for 40\% of all CPU time. The program uses SSE2 instructions (implemented via compiler intrinsics) to compare each pair of codewords in four instructions.

Feedback frequency counting. The heuristic code breaker relies heavily on these routines to count statistics on partitions. This is an intensive operation which accounts for about 20\% of all CPU time. The program uses an ASM implementation to maximize performance. See Frequency.cpp.

\section{Data Structure}

\subsection{Codeword}

\subsection{Feedback}

\subsection{Strategy Tree}

\section{Basic operations}

\subsection{Codeword comparison}

\subsection{Frequency table generation}

\subsection{Codeword set partitioning}

\section{Implementing Heuristic Strategies}

 Obvious guesses
 
\section{Implementing Optimal Strategies}
 
\section{Implementing randomized strategies}


\appendix
\section{Appendix}

\subsection{Mathematical Background}

\subsubsection{Information entropy}

entropy

\subsubsection{Equivalence relation}

See \url{http://en.wikipedia.org/wiki/Equivalence\_relation}.

\subsubsection{Permutation}

[permutation notations, properties]

[equivalence relation, equivalence class, etc.]

A \emph{permutation} is a bijection (one-to-one mapping) of a set onto itself.\footnote{
More details can be found at \url{http://en.wikipedia.org/wiki/Permutation}.}
For example, consider a set containing six elements. For convenience, label each element with an index starting from one. A possible permutation is the following:
\[
\begin{pmatrix}
1 & 2 & 3 & 4 & 5 & 6 \\
5 & 2 & 1 & 6 & 3 & 4
\end{pmatrix} ,
\]
where the first row displays the elements in the set, and the second row displays the \emph{images} of the elements under the permutation, i.e.\ the element that each element is mapped to mapped to.

The above notation for the permutation can be abbreviated into one line by only keeping the second row, which becomes $(5 2 1 6 3 4)$.

A permutation can be decomposed into a \emph{product} of disjoint \emph{cycles}, which partitions the elements of the set into parts where the elements in each part can be permuted independently to complete the overall mapping. For example, the above permutation can be written as
\[
\begin{pmatrix}
1 & 2 & 3 & 4 & 5 & 6 \\
5 & 2 & 1 & 6 & 3 & 4
\end{pmatrix} 
= (1 5 3) (2) (4 6) .
\]
It is easy to see that the cycles can be commuted and the elements in a cycle can be rotated without changing the overall permutation. Up to these differences, such decomposition is unique.

A permutation can be inverted. To find the inverse of a permutation, simply exchange the two roles in the notation and then sort the upper row. In the above example, the inverse permutation is
\[
\begin{pmatrix}
1 & 2 & 3 & 4 & 5 & 6 \\
3 & 2 & 5 & 6 & 1 & 4
\end{pmatrix} .
\]

%Two permutations of the same size can be compounded to form a composite permutation. Let @c P be the composite of permutation
% * <code>P<sub>1</sub></code> and <code>P<sub>2</sub></code>.
% * The effect of applying @c P is equivalent to first applying
% * <code>P<sub>1</sub></code> followed by applying <code>P<sub>2</sub></code>.
% * The notation to write a composite permutation can be confusing,
% * so we omit it here.

A \emph{partial permutation} on a set is a bijection between two subsets of it.\footnote{For more details, see \url{http://www.maths.qmul.ac.uk/~pjc/odds/partial.pdf}.}
For example, a partial permutation of the above (complete) permutation could be
\[
\begin{pmatrix}
1 & 2 & 3 & 4 & 5 & 6 \\
* & 2 & * & * & 3 & 4
\end{pmatrix} ,
\]
where the asterisks denote unmapped elements, sometimes known as ``holes'' of the permutation. 

It is easy to see that any partial permutation can be \emph{extended} to form a complete permutation. However such extension is not unique. For example, there are $3! = 6$ ways to extend the above partial permutation, one of which that differs from the original example could be
\[
\begin{pmatrix}
1 & 2 & 3 & 4 & 5 & 6 \\
1 & 2 & 5 & 6 & 3 & 4
\end{pmatrix} .
\]

\subsubsection{Graph isomorphism}

\subsection{Optimal strategy tree for Mastermind}

With repetition, without repetition

\subsection{Comparison of different strategies}

display a set diff of the strategy trees

\nocite{*}
\bibliographystyle{plain}
\bibliography{mastermind}

\end{document}
